\documentclass[reqno,letterpaper, onsided,10pt]{amsart}
\usepackage[dvips, foot=0.3in, left=1in, right=1in, top=1in, bottom=1in]{geometry}
\usepackage{fancyhdr,lastpage, graphicx}
\usepackage[usenames,dvipsnames]{color}
\raggedbottom
%\renewcommand{\fboxrule}{1pt}

\newcommand{\sol}[1]{\par\noindent{\bf Solution:} #1}
\renewcommand{\sol}[1]{}

\pagestyle{fancy}
\thispagestyle{fancy}
\renewcommand{\footrulewidth}{0.4pt}
\renewcommand{\headrulewidth}{0.4pt}

%\renewcommand{\arraystretch}{1.5}
\newcommand{\hwnum}{12}

\lhead{{\sc Quiz:} \hwnum}
\rhead{{\sc Page:} \thepage \,of \pageref{LastPage}}
\chead{{\sc Course:} M339J - Probability Models}

\cfoot{{\sc Instructor:} Milica \v Cudina}
%\rfoot{{\sc Semester:} Fall 2013}
\lfoot{}

% Definitions
%%%%%%%%%%%%%%
\usepackage{mydefs}
\usepackage{mythm_long}
%\input{local_def.tex}

\begin{document}

\begin{center}
{{\sc University of Texas at Austin}}
\end{center}
\vspace{1ex}

\begin{center}
\underline{\Large Quiz \#\hwnum}\\

\medskip

{\large Discrete distributions.} 
\end{center}

\bigskip
\hrule
\bigskip

{\color{blue} Provide your \underline{\bf complete solution} to the following problems. Final answers only, without appropriate justification, will receive zero points even if correct.}

\bigskip
\hrule
\bigskip

\setcounter{section}{\hwnum}

\begin{problem} ($5$ points) %spring 2021
%the Poisson distribution
{\em Source: Sample P exam, Problem \#30.}\\
An actuary has discovered that policyholders are three times as likely to file two claims as to file four claims. The number of claims filed has a Poisson distribution. Calculate the variance of the number of claims filed.

\mchoice{$1/\sqrt{3}$}{1}{$\sqrt{2}$}{2}{None of the above.}
\end{problem}

\sol{{\bf (d)}\\
Let $N$ denote the number of claims. We are given that
\begin{equation}\nonumber
  \begin{split}
    \PP[N=2] = 3\PP[N=4] \quad \Rightarrow \quad
      e^{-\lambda} \frac{\lambda^2}{2!} = 3e^{-\lambda} \frac{\lambda^4}{4!}
  \end{split}
\end{equation}
where $\lambda$ denotes the parameter of the Poisson distribution. The above equation yields that $\lambda^2=4$, so that $\lambda=2$. Since the variance of the Poisson distribution equals its parameter value, our answer is $2$.
}

\medskip

\begin{problem} ($5$ points) 
Let the number of floods in a calendar year be denoted by $N$ and
modeled using the Poisson distribution with mean 5.  We say that a
flood is ``minor'' if the damages associated with it do not exceed
\$1,000,000. Otherwise, a flood is designated as ``major''.  The
number of floods and the damages caused by individual floods are
assumed to be independent.

Assume that the probability that an observed  flood is ``major''
equals 1/5. 

Find the probability that the number of ``major'' floods is 2, 
given that the {\bf total} number of floods in that year equals 5. 
\end{problem}

\sol{
Let $N_1$ denote the r.v. which stands for the number of ``major''floods, and let $N_2$ be the number of
``minor'' floods. According to the {\em "Thinning" theorem},  $N_1$ and $N_2$ are
independent and 
\begin{equation}%\label{}
    \nonumber 
    \begin{split}
      N_1 & \sim Poisson(\frac{1}{5}\cdot 5=1), \\
      N_2 & \sim Poisson(\frac{4}{5}\cdot 5=4).  
    \end{split}
\end{equation}  
We are ready to calculate the conditional probability 
\begin{equation}%\label{}
    \nonumber 
    \begin{split}
      \PP[N_1  = 2 \, | \, N = 5] 
      & = \frac{\PP[N_1  = 2,  N = 5]}{\PP[N=5]}\\
      & = \frac{\PP[N_1  = 2,  N_1 + N_2 = 5]}{\PP[N=5]}\\
      & = \frac{\PP[N_1  = 2,  N_2 = 3]}{\PP[N=5]}. 
    \end{split}
\end{equation}
Since $N_1$ and $N_2$ are independent, this probability equals 
\begin{equation}%\label{}
    \nonumber 
    \begin{split}
     \frac{\PP[N_1  = 2]\,\PP[N_2 = 3]}{\PP[N=5]}
     & = \frac{e^{-1} \, \frac{1^2}{2!} \cdot
       e^{-4}\,\frac{4^3}{3!}}{e^{-5}\, \frac{5^5}{5!}}\\ 
     & = \frac{\frac{1^2}{2!} \cdot
       \frac{4^3}{3!}}{\frac{5^5}{5!}}\\ 
     & = \frac{4^3 \cdot 5!}{5^5 \cdot 2!\cdot 3!} = \frac{2^7}{5^4} = 0.2048.
    \end{split}
\end{equation}
Of course, we obtained the binomial conditional distribution above. This is a fact we have shown in class and you could have just used it directly. 
}

\medskip

\begin{problem} ($5$ points) %three -- spring 2021
%Poisson conditioning -> binomial
Suppose that the number $N$ of customers visiting a fast food restaurant in a given morning is Poisson with mean $20$. Assume that each customer purchases a drink with probability $3/4$, independently from other customers, and independently from the value of $N.$ Let $N_1$ be the number of customers who purchase drinks in that time interval and let $N_2$ be the number of customers that do not purchase drinks. 

What is the probability that exactly $3$ customers purchase a drink in a given morning, {\bf given} that there is a total of $10$ customers on that particular morning?
\end{problem}

\sol{
We have established in class that 
\begin{equation}\nonumber
  \begin{split}
     N_1 \, |\, N = 10 \sim Binomial(m=10, q=3/4).
  \end{split}
\end{equation}
Hence, 
\begin{equation}\nonumber
  \begin{split}
     \PP[N_1 = 3 \, | \, N = 10] = \binom{10}{3} \left(\frac{3}{4}\right)^3 \left(\frac{1}{4}\right)^7  = \frac{10 \cdot 9 \cdot 8}{3\cdot 2} \cdot \frac{3^3}{4^{10}} = 0.0030899. 
  \end{split}
\end{equation}
}

\medskip


\end{document}

%%% Local Variables:
%%% mode: latex
%%% TeX-master: t
%%% End:
