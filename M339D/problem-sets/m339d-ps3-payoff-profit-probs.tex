\documentclass[reqno,letterpaper, onsided,10pt]{amsart}
\usepackage[dvips, foot=0.3in, left=1in, right=1in, top=1in, bottom=1in]{geometry}
\usepackage{fancyhdr,lastpage,xcolor,graphicx}
\raggedbottom
%\renewcommand{\fboxrule}{1pt}

\definecolor{huntergreen}{RGB}{53, 94, 59}
\definecolor{maroon}{RGB}{128, 0, 0}

\newcommand{\sol}[1]{\par\noindent{\bf Solution:} #1}
%\renewcommand{\sol}[1]{\newpage}

\newcommand{\ccrfir}{continuously compounded, risk-free interest rate }

\pagestyle{fancy}
\thispagestyle{fancy}
\renewcommand{\footrulewidth}{0.4pt}
\renewcommand{\headrulewidth}{0.4pt}

%\renewcommand{\arraystretch}{1.5}
\newcommand{\hwnum}{3}


\lhead{{\sc PS:} \hwnum}
\rhead{{\sc Page:} \thepage \,of \pageref{LastPage}}
\chead{{\sc Course:} M339D/M389D - Intro to Financial Math}

\cfoot{{\sc Instructor:} Milica \v Cudina}
%\rfoot{{\sc Semester:} Fall 2013}
\lfoot{}

% Definitions
%%%%%%%%%%%%%%
\usepackage{mydefs}
\usepackage{mythm_long}
%\input{local_def.tex}

\begin{document}

\begin{center}
{{\sc University of Texas at Austin}}
\end{center}
\vspace{1ex}

\begin{center}
\underline{\Large Problem Set \hwnum }\\

\medskip

\underline{\large Payoff. Profit.}
\end{center}

\setcounter{section}{\hwnum}

\bigskip
\hrule
\bigskip

\subsection{Static portfolios.}
\paragraph{{\color{huntergreen} \underline{\it Step \#1}}} 
Remember the {\bf bottom-line approach} from {\it theory of interest}. 
Decide who your {\bf protagonist} is!

\paragraph{{\color{huntergreen} \underline{\it Step \#2}}} 
Set up the {\bf timeline} (on paper or mentally):


\vfill

\begin{center}
\fbox{
\begin{minipage}{30em}    
    This is how we will talk about {\bf profit}:
\begin{itemize}
\item[$\bullet$]
If {\color{huntergreen} {\bf Profit}$>0$}, 
then we call it a {\color{maroon} {\bf gain}}. 
\item[$\bullet$]
If {\color{huntergreen} {\bf Profit}$<0$}, 
then we call it a {\color{maroon} {\bf loss}}. 
\item[$\bullet$]
If {\color{huntergreen} {\bf Profit}$=0$}, 
then we say that we {\color{maroon} {\bf break even}}. 
\end{itemize}
\end{minipage}
}
\end{center}

\bigskip

\newpage

\subsection{Riskless assets}

\begin{example} {\color{purple}{\bf Investing in a zero-coupon bond}}

\end{example}

\vskip8cm

\begin{example} {\color{purple}{\bf Taking a loan}}

\end{example}

\newpage

\subsection{Risky assets}
\begin{example} {\color{purple}{\bf Outright purchase of a stock}}

\end{example}

\vskip8.5cm

\begin{problem} 
Let the current price of a non-dividend-paying stock be \$40. The \ccrfir is $0.04$. You model the distribution of the time$-1$ price of the above stock as follows:
\begin{equation} % \label{}
\nonumber
 \begin{split}
   S(1) \sim \begin{cases}
    45, & \text{with probability $1/4$}, \\
        42, & \text{with probability $1/2$}, \\
            38, & \text{with probability $1/4$}. \\
    \end{cases}
 \end{split}
\end{equation}
What is your expected profit under the above model, if you invest in one share of stock at time$-0$ and liquidate your investment at time$-1$?
\end{problem}

\sol{
  The initial cost is $S(0)$ and the payoff is $S(T)$ with $T=1$. So, the profit equals 
  \begin{equation} % \label{}
  \nonumber
   \begin{split}
      S(T) - S(0) e^{rT}.  
   \end{split}
  \end{equation}
  Thus, the expected profit equals 
  \begin{equation} % \label{}
  \nonumber
   \begin{split}
     \EE[S(T)] - S(0) e^{rT}.  
   \end{split}
  \end{equation}
According to the given model for the stock price, we have 
\begin{equation} % \label{}
\nonumber
 \begin{split}
   \EE[S(T)] = 45\left(\frac{1}{4}\right)  +  42\left(\frac{1}{2}\right) +  38 \left(\frac{1}{4}\right) = 41.75.   
 \end{split}
\end{equation}
Finally, the expected profit is 
\begin{equation} % \label{}
\nonumber
 \begin{split}
   41.75 - 40 e^{0.04} = 0.117569.   
 \end{split}
\end{equation}
}

\newpage

\paragraph{{\color{huntergreen} \underline{\it Goal}}} 
To study the payoff and the profit as {\color{maroon} {\bf functions}} 
of the {\color{maroon}{\bf final asset price}}. 

\paragraph{{\color{huntergreen} \underline{\it Introduce}}}
{\color{maroon} $s \dots$ an independent \underline{\bf argument} 
taking values in $[0, \infty)$ which will stand for the 
{\color{maroon} \bf final asset price}, i.e., it will be a "placeholder" 
for the random variable $S(T)$
}

\medskip

\newpage

\begin{problem}
 To plant and harvest 20,000 bushels of corn, Farmer Jayne incurs total aggregate
  costs totaling \$33,000. The current spot price of corn is \$1.80
  per bushel. What is the profit if the spot price is \$1.90 per
  bushel when she harvests and sells her corn?
\begin{itemize}
\item[(a)] About \$3,000 gain
\vskip0.1cm
\item[(b)] About \$3,000 loss
\vskip0.1cm
\item[(c)] About \$5,000 loss
\vskip0.1cm
\item[(d)] About \$5,000 gain
\vskip0.1cm
\item[(e)] None of the above
\end{itemize}
\end{problem}

\sol{{\bf (d)}\\
\begin{equation}%\label{}
    \nonumber
    \begin{split}
      1.90\cdot 20,000 - 33,000 = 5,000
    \end{split}
\end{equation}
}

\medskip


\end{document}

%%% Local Variables:
%%% mode: latex
%%% TeX-master: t
%%% End:
